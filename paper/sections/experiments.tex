
\section{Implementation and Experiments}
%
TODO: In this section, we briefly describe the implementation of our benchmark suite.

As described in Section \ref{sec:methods:impuation}, we define a framework that provides for each of the six implemented imputation approaches a common API with \code{fit} and \code{transform}. \code{fit} trains the imputer for the given data and cross-validates its hyperparameters and \code{transform} allows to impute missing values of the target column the imputer is trained on. For our implementation, we use \emph{tensorflow} version 2.4.1, \emph{scikit-learn} version 0.24.1, and \emph{autokeras} version 1.0.12.

For a comprehensive comparison under realistic conditions, we use the package \code{jenga}\footnote{\url{https://github.com/schelterlabs/jenga}} \citep{Jenga} to discard values. In all experiments, we spread the amount of missing values, e.g., $30\%$, over all columns. As an example, for a given missingness pattern, e.g., \emph{MAR}, we introduce $\frac{30\%}{n}$ missing values of the pattern MAR to each of the $n$ columns.

The following section presents our experimental settings.


\subsection{Experiments}

We repeatedly run experiments with the settings shown in Table \ref{tab:experiment_settings}. For each of the data sets, we randomly sampled one to-be-imputed (or target) column that remains static throughout all of our experiments (see supplementary material). To reduce the randomness in our results, we run each setting three times and report the mean values.
%
\begin{table}[h!]
	\centering
	\begin{tabular}{ll}
		\toprule
		Parameter            & Values                                     \\ \midrule
		Datasets             & 69 datasets (see supplementary material)    \\
		Imputer              & Mode, $k$-NN, Random Forest, DL, GAIN, VAE \\
		Missingness Pattern  & MCAR, MAR, MNAR                            \\
		Missingness Fraction & $1\%, 10\%, 30\%, 50\%$                      \\ \bottomrule
	\end{tabular}
	\caption{TODO.}
	\label{tab:experiment_settings}
\end{table}
%

As mentioned in Section \ref{sec:introduction}, one of our goals is to imputer's performance on two scenarios. These are described in the following sections.


\subsubsection{Fully Observed Data}
%



\subsubsection{Incomplete Data}
%


\subsection{Evaluation}
%
Was wir messen:
Imputation Accuracy und Downstream Performance

Imputation Accuracy, depending on column data type:
categorical: F1 macro
numerical: RMSE

Downstream Performance, depending on data type:
Classification: F1 macro
numerical: RMSE
