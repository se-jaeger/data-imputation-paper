%!TEX root = ../data-imputation.tex

\begin{abstract}
%
With the increasing importance and complexity of data pipelines, data quality became one of the key challenges in modern software applications. The importance of data quality has been recognized beyond the field of data engineering and database management systems (DBMS): Also, for machine learning (ML) applications, high data quality standards are crucial to ensure robust predictive performance and responsible usage of automated decision-making. One of the most frequent data quality problems is missing values. Incomplete data sets can break data pipelines and can have a devastating impact on downstream ML applications when not detected. While statisticians and, more recently, ML researchers have introduced a variety of approaches to impute missing values, comprehensive benchmarks comparing classical and modern imputation approaches under fair and realistic conditions are underrepresented. Here we aim to fill this gap. We conduct a comprehensive suite of experiments on a large number of data sets with heterogeneous data and realistic missingness conditions, comparing both novel deep learning approaches and classical ML imputation methods when either only test or train and test data are affected by missing data. Each imputation method is evaluated regarding the imputation quality and the impact imputation has on a downstream ML task. Our results provide valuable insights into the performance of a variety of imputation methods under realistic conditions. Further, they help to guide data preprocessing method selection for research as well as application.
%
	\keyFont{ \section{Keywords:} data quality, data cleaning, imputation, missing data, benchmark, MCAR, MNAR, MAR} %All article types: you may provide up to 8 keywords; at least 5 are mandatory.
\end{abstract}
