%!TEX root = ../data-imputation.tex

\section{Methods}
\label{sec:methods}
%
One of the main goals of this work is to provide a comprehensive evaluation of missing value imputation methods under realistic conditions. In particular we focus on two aspects a) a large suite of real-world data sets and tasks and b) realistic missingness patterns. The following sections describe the data sets we considered as well as the missingness patterns, followed by a detailed description of the imputation methods compared and the metrics used for the evaluation.

\subsection{Datasets}
%
We focus on a broad evaluation with several numeric datasets and tasks (regression, binary classification, and multi-class classification). The OpenML database\footnote{Website: \url{https://www.openml.org/}} contains thousands of datasets and provides an API to automatically download them. The Python package \code{scikit-learn} uses this API to retrieve the datasets and creates well-formed \code{DataFrames} that encode the tables' columns properly.

We filter available datasets as follows. To calculate the imputation performance, we need ground truth datasets that do not contain missing values. Further, especially deep learning models need sufficient data to learn their task properly. However, because we plan to run many experiments, the datasets must not be too big to keep training times feasible. For this reason, we choose datasets without missing values that contain 5 to 25 features and 3k to 100k observations. We then removed duplicated, corrupted, and Sparse ARFF\footnote{Attribute-Relation File Format} formatted datasets.

The resulting 69 datasets are composed of 21 regression, 31 binary classification, and 17 multi-class classification datasets. The supplementary material contains a detailed list of all datasets and further information, such as OpenML ID, name, and the number of observations and features.


\subsection{Missingness Patterns}
\label{sec:missingess_pattern}
Most research on missing value imputation considers three different types of missingness patterns:
%
\begin{itemize}
\item Missing completely at random (MCAR, see \autoref{tab:missingness_patterns_MCAR}): \\
Values are discarded independent of any other values
\item Missing at random (MAR, see \autoref{tab:missingness_patterns_MAR}): \\
Values in column $c$ are discarded dependent on values in another column $k\neq c$
\item Missing not at random (MNAR, see \autoref{tab:missingness_patterns_MNAR}): \\
Values in column $c$ are discarded dependent on their value in $c$
\end{itemize}
%
The most often used missingness pattern in the literature on missing value imputation is MCAR. Here the missing values are chosen independently at random. Usually the implementations of this condition draw a random number from a uniform distribution and discard a value if that random number was below the desired missingness ratio. Few studies report results on the more challenging conditions MAR and MNAR. We here aim for a realistic modelling of these missingness patterns inspired by observations in large scale real world data sets as investigated in \cite{Biessmann2018a}. We use an implementation proposed in \cite{Schelter2020a} and \cite{Jenga}, which selects two random percentiles of the values in a column, one for the lower and one for the upper bound of the value range considered. The range of the upper and lower bound depend on the desired fraction of values to be discarded. In the MAR condition, we discard values if values in a random other column fall in that percentile. In the MNAR condition we discard values in a column if the values themselves fall in that random percentile range.
%
\begin{table}
	\centering
%	\caption{
%		Examples of missingness patterns for a missingness ratio of 50\%. 	}
%	\label{tab:missingness_patterns}
%	\vspace{1em}
%	\begin{subtable}{0.3\textwidth}
\begin{minipage}{0.28\textwidth}
\centering
	\begin{tabular}{cc}
\toprule
 height &  height$_{\text{MCAR}}$ \\
\midrule
  179.0 &                     ? \\
  192.0 &                     ? \\
  189.0 &                 189.0 \\
  156.0 &                 156.0 \\
  175.0 &                     ? \\
  170.0 &                 170.0 \\
  181.0 &                     ? \\
  197.0 &                     ? \\
  156.0 &                 156.0 \\
  160.0 &                 160.0 \\
\bottomrule
\end{tabular}
\caption{
		Applying the MCAR condition to column \textit{height} discards five out of ten values independent of the height values.
	}
	\label{tab:missingness_patterns_MCAR}
\vspace{2em}
\end{minipage}
\hfill
\begin{minipage}{0.3\textwidth}
\centering
	\begin{tabular}{ccc}
\toprule
 height & gender &  height$_{\text{MAR}}$ \\
\midrule
  200.0 &      m &                    ? \\
  191.0 &      m &                    ? \\
  198.0 &      f &                198.0 \\
  155.0 &      m &                    ? \\
  206.0 &      m &                    ? \\
  152.0 &      f &                152.0 \\
  175.0 &      f &                175.0 \\
  159.0 &      m &                    ? \\
  153.0 &      f &                153.0 \\
  209.0 &      m &                209.0 \\
\bottomrule
\end{tabular}
\caption{In the MAR condition \textit{height} values are discarded dependent on values in another column,  here \textit{gender}. All discarded \textit{height} values correspond to rows in which \textit{gender} was \textit{male}.
}
	\label{tab:missingness_patterns_MAR}
\end{minipage}
\hfill
\begin{minipage}{0.28\textwidth}
\centering
	\begin{tabular}{cc}
\toprule
 height &  height$_{\text{MNAR}}$ \\
\midrule
  154.0 &                     ? \\
  181.0 &                 181.0 \\
  207.0 &                 207.0 \\
  194.0 &                 194.0 \\
  153.0 &                     ? \\
  156.0 &                     ? \\
  198.0 &                 198.0 \\
  185.0 &                 185.0 \\
  155.0 &                     ? \\
  164.0 &                     ? \\
\bottomrule
\end{tabular}
\caption{In the MNAR condition \textit{height} values are discarded dependent on the actual \textit{height} values. All discarded values correspond to small \textit{height} values.
}
	\label{tab:missingness_patterns_MNAR}
\vspace{1em}
\end{minipage}

\end{table}

\subsection{Data Preprocessing}
\felix{Here we can describe the encoding steps independently of the imputation methods}

\subsection{Imputation Methods}
\label{sec:methods:impuation}
%
\felix{
\begin{itemize}
\item subsubsections are too deeply nested, I'd replace them by paragraphs but the journal style doesn't like that ...
\item we should avoid the term imputer, even if it makes sense for the API, we should call it imputation method
\item we should stick to American (e.g. normalize) or British English (e.g. normalise)
\item As for the notation, it's common to follow the convention: $\vec{x}\in\R^d$ for $d$-dimensional vectors, $\vec{X}\in\R^{n\times d}$ for matrices with n rows and d columns,  scalars are denoted by $x$, single coefficients can be denoted as $\vec{x}_i$ and scalars in row i and column j can be denoted as $\vec{X}_{i,j}$, but the latter two are not really a convention. 
\item most importantly: if we don't need the formulars/math notation to make a point, like derive anything or present an objective/proof, we might as well discard it entirely.
\item We should have a table with all hyperparameters in one place
\item the encoding should be in one place, at least for discriminative and generative models
\end{itemize}
}
In this section, we describe our six single imputation methods. The overall goal of an imputer is to train a model on $X = [X_1, X_2, ..., X_{i-1}, X_{i+1}, ..., X_n]$, where $n$ is the number of features and $X_i$ the to-be-imputed (or target) column. \felix{the following statement is not true for all imputation methods} Consequently, if there are $m$ columns with missing values, we need to train $m$ imputation models. However, to abstract this and other crucial steps, such as encode, normalise, and decode the data and cross-validate hyperparameters of each imputation method, we define a framework implemented by all of our imputer implementations. As described in Section \ref{sec:introduction}, missing values can break ML pipelines. By default, we substitute missing values with column-wise mean/mode to prevent the framework from failing.


\subsubsection{Simple Imputer}
%
Our \code{Simple Imputer} uses the column-wise \code{mean} for numerical or \code{mode}, i.e., the most frequent value,  for categorical columns to fill missing values.

\subsubsection{KNN Imputation}
%
A popular ML imputation baseline is KNN imputation, also known as Hot-Deck imputation~\citep{Batista2003}. \felix{isn't the encoding the same for most methods? if so, we should place that somewhere else, before the methods, and highlight the different steps in the pipeline in a flow chart} We encode categorical features as one-hot columns and normalize the data by rescaling it to zero mean and unit variance. The hyperparameter $k \in \{1, 3, 5\}$ is optimized by 5-fold cross-validated grid-search.
Depending on the target column datatype (categorical or numerical) we use the \code{scikit-learn} implementation \code{KNeighborsClassifier} or \code{KNeighborsRegressor}, respectively.

\subsubsection{Random Forest Imputation}
%
Another popular ML based imputation method is based on Random Forests. Also here we encode categorical features as one-hot columns and normalize the data by rescaling it to zero mean and unit variance
Depending on the target column we use either the \code{RandomForestClassifier} or \code{RandomForestRegressor} and optimize the hyperparameter $n\_estimators \in \{10, 50, 100\}$ optimized by 5-fold cross-validated grid-search.

\subsubsection{Discriminate Deep Learning Imputation}
%
Often simple deep learning models can achieve good imputation results~\cite{Biessmann2018}. To easily optimize the model's architecture, we use the AutoML\footnote{"automated machine learning (AutoML) [...] automatically set [the model's] hyperparameters to optimize performance" \cite{AutoML}} library \code{autokeras} \citep{AutoKeras} to build our \emph{Deep Learning Imputer}.
%
For categorical columns, we use AutoKeras' \code{StructuredDataClassifier} and for numerical columns \code{StructuredDataRegressor}. Both classes take care of properly encode the data and optimize the model's architecture and hyperparameters. To reduce the training time, we change the maximum number of trials to $50$, which means \code{autokeras} tries 50 different model architecture and hyperparameter combinations, and the maximal number or of $epochs$ (\code{autokeras} uses early stopping) to 50.


\subsubsection{Generative Deep Learning Imputation}
%
All of the above approaches essentially follow the ideas known in the statistics literature as multiple imputation with chained equations (MICE) \citep{Little} or as {\em fully conditional specification} \citep{vanBuuren2018}: a discriminative Model is trained on all but one column as features and the remaining column as the target variable. This approach has the advantage to be applicable to any supervised learning method, but it has the decisive disadvantage that for each to-be-imputed column a new model has to be trained. Generative approaches are different in that they train just one model for an entire table. All matrix factorization based approaches such as \citep{Troyanskaya2001,Koren2009,Mazumder2010} can be thought of as an example of generative models for imputation. We do not consider those linear generative models here as they have been shown to be outperformed by the above mentioned methods and focus on deep learning variants of generative models only. 

Generative Deep Learning methods can be broadly categorized in to classes, (variational) autoencoders (VAE)~\citep{VAE}\footnote{We focus on probabilistic autoencoders here as there are more imputation methods available for VAEs} and generative adversarial networks (GAN) In the following we shortly highlight some representative imputation methods based on either of these two and describe the implementation used in our experiments. 

\paragraph{Variational Autoencoder (VAE) Imputation}
VAEs learn to encode their input into a distribution over the latent space and decode by sampling from this distribution \citep{VAE}. Imputation methods based on this type of generative model include \citep{HIVAE, VAE_for_genomic_data, VAEM}. Rather than comparing all existing implementations we here focus on the original VAE imputer for the sake of comparability with other approaches. We treat the number of hidden layers $n\_hidden\_layers$ for encoder and decoder as hyperparameters\felix{Which values were chosen?}. The size of these layers, i.e., the number of neurons, are fix and set relatively to the input dimension, i.e., the data set's number of columns\felix{Why? Can we motivate that decision?}. The dimensionality of the hidden layer one is $50\%$, hidden layer two $30\%$, and latent space is $20\%$ of the input dimension.
\felix{not sure this sentence is grammatically correct?} To encode the categorical columns and replace missing values with random noise \felix{what noise? uniform, normal? what mean/std?} (\cite{CaminoVAE} already present good results with this approach), we use the same preprocessing strategy as for the \emph{GAIN Imputer}.
\felix{didn't we also have mean/0 imputation instead of noise?}
To optimize the hyperparameter $n\_hidden\_layers \in \{0, 1, 2\}$, we use 3-fold cross-validated grid-search.

\paragraph{Generative Adversarial Network (GAN) Imputation}
GANs consist of two parts \citep{GAN}: a generator and a discriminator. In an adversarial process, the generator learns to generate samples that are as close as possible to the data distribution, and the discriminator learns to distinguish whether an example is true or generated. Imputation approaches based on GANs include \citep{GAIN, VIGAN, MisGAN}. 
Here we employ one of the most popular approaches of GAN based imputation, Generative Adversarial Imputation Nets (GAIN) \cite{GAIN}. \felix{if this is the only place where we used math notation, we might consider rephrasing this, none of the other methods is described in such detail.} GAIN takes as input some data $X$, encodes categorical columns as numbers from $0$ to $n-1$, where $n$ is the number of categories, and calculates a binary mask matrix $M$ that represents missing values. To normalize $X$ into $\bar{X}$, it first scales the data min-max ($0, 1$) and second replaces missing values with random uniform noise ($\{0, 0.01\}$). The generator learns to output $\hat{X}$, where initially missing values are replaced, based on its input $\bar{X}$ and $M$. The discriminator learns to reconstruct the mask $M$ based on generator's output $\hat{X}$ and a hint matrix $H$, using the hyperparameter $hint\_rate$, that provides the discriminator with information about $M$.

GAIN is optimized by minimizing the sum of the generator's loss and the with $\alpha$ weighted discriminator's loss, see \cite{GAIN} for details. Besides the learning rates, for the generator and discriminator, GAIN introduces two new hyperparameters we optimize $hint\_rate$ and $\alpha$. For this, we use 3-fold cross-validated grid-search of: $generator\_learning\_rate \in \{0.0001, 0.0005\}$, $discriminator\_learning\_rate \in \{0.00001, 0.00005\}$, $\alpha \in \{1, 10\}$, and $hint\_rate \in \{0.7, 0.9\}$.


The \emph{GAIN Imputer} as well as the \emph{VAE Imputer} are trained using Adam optimizer with default hyperparameters for max $50$ epochs (early stopping) and batch size of $64$.


\subsection{Evaluation Metrics}
%
To evaluate our experiments, we use two metrics. For regression downstream tasks or numerical imputation, we use the $RMSE$, shown in Equation \ref{eq:RMSE}, and for classification downstream tasks or categorical imputation the $macro\ F1$, shown in Equation \ref{eq:F1}.
%
\begin{equation}
	RMSE = \sqrt{\frac{1}{N} \sum_{i = 0}^{N} (x_i - \hat{x_i})^2}
	\label{eq:RMSE}
\end{equation}
%
\begin{equation}
	macro\ F1 = \sum_{i = 0}^{N} F1_i\text{, where }F1 = \frac{TP}{TP + \frac{1}{2}(FP + FN)}
	\label{eq:F1}
\end{equation}
%
Because $RMSE$ measures an error, the lower its value the better the performance. On the other hand, $F1$ (we use $macro\ F1$ and $F1$ synonym) calculates a score and, therefore, greater $F1$ means better performance.
