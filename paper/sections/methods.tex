

\section{TODO: Methods}
%
TODO: Übergang

\subsection{Datasets}
%
We focus on a broad evaluation with several numeric datasets and the most widespread tasks: regression, binary classification, and multi-class classification. The OpenML database\footnote{\url{https://www.openml.org/}} contains many thousand datasets that can directly be retrieved as \code{pandas DataFrames}\footnote{TODO Link + it is quasi standard..} that encode the columns' datatypes properly, which means categorical features are of \code{pandas dtype} \code{category}.

Especially deep learning models need sufficient data to learn their task properly. However, because we plan to run many experiments, the datasets must not be too big to keep training times feasible. This is why we choose datasets without missing values that contain 5 to 25 features and 3k to 100k observations. We then removed duplicated, corrupted, and Sparse ARFF\footnote{Attribute-Relation File Format} formatted datasets.

The resulting 70 datasets are composed of 22 regression, 31 binary classification, and 17 multi-class classification datasets. The supplementary material contains a detailed list of all datasets and further information, such as OpenML ID, name, and the number of observations and features.


\subsection{TODO: Jenga}
%


\subsection{Imputation Methods}
%
In this section, we describe our six single imputation methods. The overall goal of an imputer is to train a model on $X = [X_1, X_2, ..., X_{i-1}, X_{i+1}, ..., X_n]$, where $n$ is the number of features and $X_i$ the to-be-imputed column. To abstract crucial steps, such as encode, normalize, and decode the data and cross-validate the imputer's hyperparameters, we define a common framework inspired by \emph{scikit-learn}\footnote{TODO Link}.

TODO: irgendwo muss hin, dass wir mit means auffüllen, wenn zur trinings time missing values auftauchen...


\subsubsection{Simple Imputer}
%
Our \code{Simple Imputer} uses the column-wise \code{mean} for numerical or \code{mode}, i.e., the most frequent value,  for categorical columns to fill missing values.


\subsubsection{Machine Learning Imputer}
%
We use two common imputation methods as representatives: \emph{K-NN Imputer} and \emph{Random Forest Imputer}. Both encode categorical features as one-hot columns and normalize the data by rescaling it to zero mean and unit variance. The imputer's hyperparameters are optimized by 5-fold cross-validated grid-search.

For our K-NN Imputer, we use, depending on the target columns' datatype (categorical or numerical), \code{scikit-learn}'s \code{KNeighborsClassifier} or \code{KNeighborsRegressor} and optimizes the $n\_neighbors \in \{1, 3, 5\}$  hyperparameter.

Similarly, the Random Forest Imputer uses the \code{RandomForestClassifier} or \code{RandomForestRegressor} and optimizes the hyperparameter $n\_estimators \in \{10, 50, 100\}$.


\subsubsection{Deep Learning Imputer}
%
Already very simple deep learning models can achieve good imputation results (TODO: cite). To easily optimize the model's architecture, we use the AutoML\footnote{"automated machine learning (AutoML) [...] automatically set [the model's] hyperparameters to optimize performance" \cite{AutoML}} library \code{autokeras} \cite{AutoKeras} to build our \emph{Deep Learning Imputer}.

For categorical columns, we use AutoKeras' \code{StructuredDataClassifier} and for numerical columns \code{StructuredDataRegressor}. Both classes take care of properly encode the data and optimize the model's architecture and hyperparameters. To reduce the training time, we change the maximum number of trials to $50$, which means \code{autokeras} tries 50 different model architecture and hyperparameter combinations, and the maximal number or of $epochs$ (\code{autokeras} uses early stopping) to 50.


\subsubsection{Generative Imputer}
%
Several generative models are successfully applied to data imputation, especially variational autoencoders (VAE) (TODO: cites), and generative adversarial networks (GAN) (TODO: cites). Autoencoders in general, learn to encode their input into an latent representation and, at the same time, to decode this latent representation back into the original feature space. In contrast to this, VAEs learn to decode their input into a distribution over the latent space and decode a sample from this distribution, which makes them generative.



\subsection{Evaluation Dimensions}
%

Was wir messen:
Imputation Accuracy und Downstream Performance

Imputation Accuracy, depending on column data type:
categorical: F1 (TODO: macro/avg?)
numerical: RMSE (TODO`?)

Downstream Performance, depending on data type:
Classification: F1(TODO: macro/avg?)
numerical: RMSE (TODO?)
